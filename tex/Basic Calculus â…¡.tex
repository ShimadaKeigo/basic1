\documentclass{jsarticle}
\usepackage{color}%use color
\usepackage{ascmac}%use itembox
\usepackage[all]{xy}%use itembox
\usepackage{amsmath,amsfonts,amsthm,amssymb,amscd}%use symbol
\usepackage[all]{xy}%use table of contents
\usepackage{fancyhdr}%use appearance
\usepackage[dvipdfmx]{graphicx}
\pagestyle{fancy}
\lhead{微分積分学基礎I\hspace{-.1em}I}
\rhead{\leftmark}
\cfoot{--\ \thepage\ --}
\title{微分積分学基礎I\hspace{-.1em}I}
\author{$\textit{lecture}$:Y.Hijikata}
\begin{document}
\maketitle
\tableofcontents
\clearpage
\section{重積分:2重積分の基礎}
\clearpage
\section{重積分:2重積分の計算と累次積分}
\clearpage
\section{重積分:2重積分の変数変換}
\clearpage
\section{広義積分と曲面積}
\clearpage
\section{1階常微分方程式:変数分離形}
・微分方程式とは?\\
\begin{eqnarray}
\lefteqn{y=f(x)}\\
\lefteqn{y' =\frac{df(x)}{dx}}
\end{eqnarray}
としたとき、
\begin{itembox}[l]{微分方程式(1階微分方程式)}
\begin{eqnarray}
y'+P(x)y=Q(x)
(P(x),Q(x)は既知の関数)
\end{eqnarray}
\end{itembox}
のような$y',y''$を含む方程式を微分方程式と呼ぶ。
\begin{itembox}[l]{微分方程式の分類}
\begin{quote}
\begin{itemize}
\item {偏微分方程式を含まない} $\longrightarrow$ {常微分方程式}
\item {偏微分方程式を含む} $\longrightarrow$ {偏微分方程式}
\item {2階以上の導関数を含まない} $\longrightarrow$ {1階微分方程式}
\item {3階以上の導関数を含まない} $\longrightarrow$ {2階微分方程式}
\end{itemize}
\end{quote}
\end{itembox}
・変数分離形
\begin{eqnarray}
g(y)dy=f(x)dx
\end{eqnarray}
の形で表すことのできる微分方程式を変数分離形という。\\
・変数分離形の解\\
不定積分
\begin{eqnarray}
\int g(y)dy = \int f(x)dx
\end{eqnarray}
から微分方程式の解を求める。
\begin{itembox}[l]{例題1}
次の微分方程式を解け。
\begin{eqnarray}
(1) \quad x\frac{dy}{dx}=-y\\
(2) \quad \frac{dy}{dx}=-2xy
\end{eqnarray}
\end{itembox}
解)\\
(1)
\begin{eqnarray}
\frac{dy}{y}=-\frac{dx}{x}\\
\int \frac{dy}{y}=\int -\frac{dx}{x}\\
\log |y| = -\log |x| +C_1\\
|y|=e^{-\log|x|+C_1}
= e^{C_1}e^{\log \frac{1}{|x|}}\\
y=\pm e^{C_1}\frac{1}{x}
\end{eqnarray}
ここで、$C=\pm e^{C_1} $とおくと、\\
\begin{eqnarray}
y=\frac{C}{x}
\end{eqnarray}
(2)
\begin{eqnarray}
\int \frac{dy}{y}=\int -2xdx\\
\log |y| = -x^2+C_1\\
|y|=e^{-x^2+C_1}\\
y=\pm e^{C_1}e^{-x^2}
\end{eqnarray}
ここで、$C = \pm e^{C_1}$とおくと、
\begin{eqnarray}
y=Ce^{-x^2}
\end{eqnarray}
\begin{itembox}[l]{演習1}
次の微分方程式を解け。
\begin{eqnarray}
&(1) \quad y\frac{dy}{dx}=x\\
&(2) \quad \frac{dy}{dx}-y=1
\end{eqnarray}
\end{itembox}
解)\\
(1)
\begin{eqnarray}
\int ydy=\int xdx\\
\frac{1}{2}y^2=\frac{1}{2}x^2+C_1\\
y=\pm \sqrt{x^2+C}
\end{eqnarray}
(2)
\begin{eqnarray}
\int \frac{dy}{y+1}=\int dx\\
\log|y+1|=x+C_1\\
|y+1|=e^{x+C_1}\\
y+1=\pm e^{C_1}e^x\\
y=Ce^x-1\\
(C=\pm e^{C_1})
\end{eqnarray}
\clearpage
\section{1階常微分方程式:同次形}
\begin{itembox}[l]{同次形}
\begin{eqnarray}
\frac{dy}{dx}=f\left(\frac{y}{x}\right)
\end{eqnarray}
の形で表すことのできる微分方程式を同次形という。
\end{itembox}
・同次形の解法
\begin{eqnarray}
\frac{y}{x}=u
\end{eqnarray}
とおけば、
\begin{eqnarray}
y=ux\\
\frac{dy}{dx}=u+x\frac{du}{dx}
\end{eqnarray}
したがって、
\begin{eqnarray}
u+x\frac{du}{dx}=f(u)
\end{eqnarray}
また、$f(u)\neq u$なら、
\begin{eqnarray}
\frac{du}{f(u)-u}=\frac{dx}{x}
\end{eqnarray}
と書き換えられるが、これは変数分離形である。
\begin{eqnarray}
\int \frac{du}{f(u)-u}=\int \frac{dx}{x}=\log |x| +C
\end{eqnarray}
一方、$f(u)=u$の場合、
\begin{eqnarray}
\frac{dy}{dx}=\frac{y}{x}\\
\frac{dy}{y}=\frac{dx}{x}
\end{eqnarray}
であり、これも変数分離形。\\
また、$u$が定数$m$の解を持つとき、
\begin{eqnarray}
u=\frac{y}{x}=m\\
y=mx \\
\frac{dy}{dx}=m=f(u)\left. \right|_{u=m}=f(m)
\end{eqnarray}
となるので、$f(m)=m$のときの解は、
\begin{eqnarray}
y=mx
\end{eqnarray}
である。\footnote{このときの解を特異解と呼ぶ。}\\
\begin{itembox}[l]{例題1}
次の微分方程式を解け。
\begin{eqnarray}
\frac{dy}{dx}=\frac{x^2+y^2}{2xy}
\end{eqnarray}
\end{itembox}
解)
\begin{eqnarray}
\frac{dy}{dx}=\frac{1+\left(\frac{y}{x}\right)^2}{2\cdot\frac{y}{x}}=\frac{1+u^2}{2u}=u+x\frac{du}{dx}\\
x\frac{du}{dx}=\frac{1-u^2}{2u}\\
\int \frac{2u}{1-u^2}du = \int \frac{dx}{x}\\
-\log |1-u^2| = \log |x|  + C\\
\log|x(1-u^2)|=C
\end{eqnarray}
もとの$x,y$を代入して整理すると、
\begin{eqnarray}
x^2-y^2=Cx
\end{eqnarray}
特異解は
\begin{eqnarray}
f(m)=\frac{1+m^2}{2m}=m\\
m=\pm 1\\
y= \pm x
\end{eqnarray}
\begin{itembox}[l]{演習1}
次の微分方程式を解け。
\begin{eqnarray}
(x^2-y^2)\frac{dy}{dx}=2xy
\end{eqnarray}
\end{itembox}
解)
\begin{eqnarray}
\frac{dy}{dx}=\frac{2\cdot\left(\frac{y}{x}\right)}{1-\left(\frac{y}{x}\right)^2}=\frac{2u}{1-u^2}\\
=u+x\frac{du}{dx}\\
x\frac{du}{dx}=\frac{u+u^3}{1-u^2}\\
\frac{1-u^2}{u(1+u^2)}du=\frac{dx}{x}\\
\left(\frac{1}{u}-\frac{2u}{1+u^2}\right)du=\frac{dx}{x}\\
\int \left(\frac{1}{u}-\frac{2u}{1+u^2}\right)du=\int \frac{dx}{x}\\
\log |u|-\log(1+u^2) = \log|x| +C
\end{eqnarray}
ここで
\begin{eqnarray}
\frac{u}{1+u^2}=\pm e^Cx\\
u=\frac{y}{x}
\end{eqnarray}
を代入して整理すると、
\begin{eqnarray}
x^2+y^2=Cy\\
x^2+\left(y-\frac{C}{2}\right)^2=\left(\frac{C}{2}\right)^2
\end{eqnarray}
特異解は、
\begin{eqnarray}
f(m)=\frac{2m}{1-m^2}\\
=m-m(1+m^2)=0\\
\Rightarrow m=0\\
\therefore y=0
\end{eqnarray}
・同次形に帰着できる場合
\begin{itembox}[l]{例題2}
次の微分方程式を解け。
\begin{eqnarray}
\frac{dy}{dx}=\frac{2x-y-1}{2y-x-1}
\end{eqnarray}
\end{itembox}
解)\\
連立方程式
\begin{displaymath}
\left\{
\begin{array}{l}
2x-y-1=0 \\
2y-x-1=0
\end{array}
\right.
\end{displaymath}
すなわち、
\begin{displaymath}
\left\{
\begin{array}{l}
2x-y=1 \\
2y-x=1
\end{array}
\right.
\end{displaymath}
この解は$x=1,y=1$なので、$\xi = x-1,\eta = y-1$と変数変換すれば、
\begin{displaymath}
\left\{
\begin{array}{l}
2x-y-1=2\xi - \eta  \\
2y-x-1=2\eta - \xi
\end{array}
\right.
\end{displaymath}
\begin{eqnarray}
\frac{d\eta}{d\xi}=\frac{d\eta}{dx}\cdot \frac{dx}{d\xi} =\frac{d\eta}{dx}\cdot 1=\frac{d\eta}{dx}\cdot \frac{dy}{d\eta}=\frac{dy}{dx}\\
\frac{d\eta}{d\xi}=\frac{2\xi-\eta}{2\eta-\xi}=\frac{2-\frac{\eta}{\xi}}{2\frac{\eta}{\xi}-1}
\end{eqnarray}
となり、これは同次形である。ここで、
\begin{eqnarray}
u=\frac{\eta}{\xi}
\end{eqnarray}
とおくと、
\begin{eqnarray}
u+\xi\frac{du}{d\xi}=\frac{2-u}{2u-1}\\
\xi \frac{du}{d\xi}=\frac{2-u}{2u-1}-u=\frac{-2(u^2-1)}{2u-1}\\
-\frac{2u-1}{2(u^2-1)}du=-\frac{1}{2}\left(\frac{2u}{u^2-1}-\frac{1}{u^2-1}\right)du=\frac{d\xi}{\xi}\\
\int \left(\frac{2u}{u^2-1}-\frac{1}{u^2-1}\right)du=-2\int \frac{d\xi}{\xi}\\
\log |u^2-1|-\frac{1}{2}\log \left|\frac{u-1}{u+1}\right|=-2\log|\xi|+C_1\\
\log \left|\frac{(u+1)(u-1)(u+1)^{\frac{1}{2}}}{(u-1)^{\frac{1}{2}}}\right|+2\log |\xi|=C_1\\
\log|(u+1)^3(u-1)\xi ^4|=2C_1
\end{eqnarray}
ここで、
\begin{eqnarray}
u=\frac{\eta}{\xi}
\end{eqnarray}
により、
\begin{eqnarray}
(\xi-\eta)(\xi+\eta)^3=C_2
\end{eqnarray}
一方、特異解は、
\begin{eqnarray}
m=\frac{2-m}{2m-1}
\end{eqnarray}
により、$m=\pm 1$。よって、
\begin{eqnarray}
\eta = \pm \xi (\because \eta = m\xi)
\end{eqnarray}
ただし、これもまた、XXXの解である。
従って、
\begin{eqnarray}
(\xi - \eta)(\xi + \eta)^3 = C (C\supset 0) 
\end{eqnarray}
とかける。
\begin{displaymath}
\left\{
\begin{array}{l}
\xi=x-1 \\
\eta=y-1
\end{array}
\right.
\end{displaymath}
を代入すると、
\begin{eqnarray}
(x-y)(x+y-2)^3 = C 
\end{eqnarray}
\begin{itembox}[l]{演習2}
次の微分方程式を解け。
\begin{eqnarray}
(3x+y-2)+(x+y)\frac{dy}{dx}=0
\end{eqnarray}
\end{itembox}
解)
\begin{eqnarray}
\frac{dy}{dx}=-\frac{3x+y-2}{x+y}
\end{eqnarray}
\begin{displaymath}
\left\{
\begin{array}{l}
3x+y=2 \\
x+y=0
\end{array}
\right.
\left\{
\begin{array}{l}
x=1  \\
y=1
\end{array}
\right.
\left\{
\begin{array}{l}
\xi =x-1  \\
\eta = y-1
\end{array}
\right.
\end{displaymath}
\begin{eqnarray}
3x+y-2=3(\xi +1)+(\eta -1)-2=3\xi+\eta \\
x+y=(\xi+1)+(\eta-1)=\xi + \eta\\
\frac{dy}{dx}=\frac{d\eta}{d\xi}=-\frac{3\xi+\eta}{\xi+\eta}=-\frac{3+u}{1+u}\\
=u+\xi \frac{du}{d\xi}\\
\xi \frac{du}{d\xi}=-\frac{3+u}{1+u}-u
=-\frac{u^2+2u+3}{u+1}\\
\int \frac{u+1}{u^2+2u+3}du = -\int \frac{1}{\xi} d\xi \\
\frac{1}{2}\log (u^2+2u+3)=-\log |\xi| + C\\
\log \sqrt{u^2+2u+3} |\xi| =C\\
(u^2+2u+3)\xi ^2 = C
\end{eqnarray}
特異解は、
\begin{eqnarray}
-\frac{3+m}{1+m}=m\\
m^2+2m+3=0
\end{eqnarray}
より、上の解に$C=0$として含まれる。
\begin{eqnarray}
\eta ^2+2\eta \xi +3\xi ^2 = C\\
(y+1)^2+2(y+1)(x-1)+3(x-1)^2\\
=(y^2+2y+1)+2(xy-x-y-1)+3(x^2-2x+1)=C\\
\therefore 3x^2+2xy+y^2-4x = C
\end{eqnarray}
\clearpage
\section{1階微分常微分方程式}
\begin{itembox}[l]{同次形と非同次形}
(同次形)\quad $y'+p(x)y=0$\\
(非同次形) $y'+p(x)y=q(x)$\\
※前回の"同次形"とは意味が違うので注意
\end{itembox}
\begin{itembox}[l]{線形と非線形}
(線形)\quad $y'+p(x)y=q(x)$\\
(非線形)$y'+p(x,y)y=q(x,y)$
※本講義では線形の場合のみ取り扱う
\end{itembox}
・同次形の解法$\rightarrow$変数分離法を用いる。\\
\begin{eqnarray}
y'+p(x)y=0\\
\frac{dy}{y}=-p(x)dx \\
\int \frac{dy}{y}=-\int p(x)dx \\
\log |y| = -\int p(x)dx +C\\
y=Ce^{-\int p(x)dx +C}
\end{eqnarray}
・非同次形の解法$\rightarrow$定数変化法を用いる。\\
\begin{eqnarray}
y'+p(x)y=q(x)
\end{eqnarray}
同次形の解
\begin{eqnarray}
y=Ce^{-\int p(x)dx +C}
\end{eqnarray}
の$C$を関数$C(x)$と仮定し、XXX式の解となるように$C(x)$と仮定し、XXX式の解となるように$C(x)$を決める。(定数変化法)
\begin{eqnarray}
y=Ce^{-\int p(x)dx +C}\\
=Ce^C\cdot e^{-\int p(x)dx}\\
C(x)e^{-\int p(x)dx}\\
=y'+p(x)y=\frac{d}{dx}\left(C(x)e^{-\int p(x)dx}\right)+p(x)C(x)e^{-\int p(x)dx}\\
=C(x)\frac{d}{dx}\left(e^{-\int p(x)dx}\right)+C'(x)e^{-\int p(x)dx}+p(x)C(x)e^{-\int p(x)dx}\\
=-C(x)p(x)e^{-\int p(x)dx}+C'(x)e^{-\int p(x)dx}+p(x)C(x)e^{-\int p(x)dx}\\
=C'(x)e^{-\int p(x)dx}\\
=q(x)\\
C'(x)=q(x)e^{\int p(x)dx}\\
C(x)=\int q(x)e^{\int p(x)dx}dx + C\\
\therefore y=e^{-\int p(x)dx}\left(\int q(x)e^{\int p(x)dx}dx + C\right)
\end{eqnarray}
\begin{itembox}[l]{例題1}
次の微分方程式を解け。
\begin{eqnarray}
(1)y'-y=x\\
(2)xy'+y=\cos x
\end{eqnarray}
\end{itembox}
解)\\
(1)
\begin{eqnarray}
p(x)=-1,q(x)=x\\
\int p(x)dx =-x\\
y=e^x\left(\int xe^{-x}dx+C\right)\\
=e^x\left(-xe^{-x}+\int e^{-x}dx+C\right)\\
=e^x(-xe^{-x}-e^{-x}+C)
=-x-1+Ce^x
\end{eqnarray}
(2)\\
与式を変形して、
\begin{eqnarray}
y'+\frac{1}{x}y=\frac{\cos x}{x}\\
\int p(x)dx=\int \frac{dx}{x} =\log|x|\\
q(x)=\frac{\cos x}{x}\\
y=e^{-\log |x|}\left(\int \frac{\cos x}{x}e^{\log|x|}dx+C\right)\\
=\frac{1}{x}\left(\int \cos xdx+C\right)\\
=\frac{\sin x}{x}+\frac{C}{x}
\end{eqnarray}
\begin{itembox}[l]{演習1}
次の微分方程式を解け。
\begin{eqnarray}
(1)y'+y=x\\
(2)y'\sin x+y\cos x =1
\end{eqnarray}
\end{itembox}
解)\\
(1)
\begin{eqnarray}
p(x)=1,q(x)=x\\
\int p(x)dx= x\\
y=e^{-x}\left(\int xe^xdx+C\right)\\
=e^{-x}(xe^x-e^x+C)\\
=x+Ce^{-x}-1
\end{eqnarray}
(2)
\begin{eqnarray}
y'+\frac{y}{\tan x}=\frac{1}{\sin x}\\
p(x)=\frac{1}{\tan x},q(x)=\frac{1}{\sin x}\\
\int p(x)dx =\int \frac{\cos x}{\sin x}dx\\
=\log |\sin x|\\
y=e^{-\log|\sin x|}\left(\int \frac{1}{\sin x}e^{\log |\sin x|}dx+C\right)\\
=\frac{x}{\sin x}+\frac{C}{\sin x}
\end{eqnarray}
\begin{itembox}[l]{演習2}
RL直列回路に対し、\\
$t=0$のとき$v(t)=E$の直流電圧を与えた。\\
$t<0$のとき$i(t)=0$だったとして、$i(t)$を求めよ。
\end{itembox}
解)\\
\begin{eqnarray}
L\frac{di(t)}{dt}+Ri(t)=E
\end{eqnarray}
の解を求める。
\begin{eqnarray}
i'+\frac{R}{L}i=\frac{E}{L}\\
\int \frac{R}{L}dt=\frac{R}{L}t\\
i=e^{-\frac{R}{L}t}\left(\int \frac{E}{L}e^{\frac{R}{L}}dt+C\right)\\
=e^{-\frac{R}{L}t}\left(\frac{E}{R}e^{\frac{R}{L}}t+C\right)\\
=\frac{E}{R}+Ce^{-\frac{R}{L}t}\\
i(0)=\frac{E}{R}+C=0
\end{eqnarray}
より、
\begin{eqnarray}
C=-\frac{E}{R}\\
i(t)=\frac{E}{R}\left(1-e^{-\frac{R}{L}t}\right)
\end{eqnarray}
\clearpage
\section{2階以上の線形微分方程式:定係数の同次形}
・n階の微分方程式\\
$y$の$n$階微分を$y^{(n)}$で表した時、
\begin{eqnarray}
y^{(n)}+p_1q^{(n-1)}+…+p_n(x)y=q(x)
\end{eqnarray}
を$n$階線形微分方程式と呼ぶ。\\
また$q(x)=0$の場合を同次形、$q(x)\neq 0$の場合を非同次形と呼ぶ。\\
・重ね合わせの原理\\
$y_1(x),y_2(x),…,y_n(x)$を同次方程式の特殊解とすれば、\\
$A_1y_1(x)+A_2y_2(x)+…+A_my_m(x)+…$も方程式の解である。\\
一次結合された上式を一般解(任意の解)という。
\begin{itembox}[l]{例題1}
$n=m=2$の場合において、重ね合わせの原理を証明せよ。
\end{itembox}
解)
\begin{eqnarray}
y''+p_1y'+p_2y=0
\end{eqnarray}
に
\begin{eqnarray}
y=A_1y_1+A_2y_2
\end{eqnarray}
を代入することにより、
\begin{eqnarray}
(A_1y_1+A_2y_2)''+p_1(A_1y_1+A_2y_2)'+p_2(A_1y_1+A_2y_2)\\
=A_1(y_1''+p_1y_1'+p_2y_1)+A_2(y_2''+p_1y_2'+p_2y_2)\\
=A_1\cdot 0+A_2\cdot 0 =0
\end{eqnarray}
・定数係数の2階線形同次方程式の解法
\begin{eqnarray}
\frac{d^2y}{dx^2}+a_1\frac{dy}{dx}+a_2y=0
\end{eqnarray}
($a_1,a_2$は定数)\\
の解を求める。$\frac{d}{dx}$を変数$s$とおくと、
\begin{eqnarray}
\left(\frac{d^2}{dx^2}+a_1\frac{d}{dx}+a_2\right)y=0\\
(s^2+a_1s+a_2)y=0
\end{eqnarray}
$\forall y$で成立するので、
\begin{eqnarray}
s^2+a_1s+a_2=0
\end{eqnarray}
固有方程式の解を$\lambda,\mu$とすると、
\begin{eqnarray}
s^2+a_1s+a_2=(s-\lambda)(s-\mu)=0
\end{eqnarray}
XXは
\begin{eqnarray}
\left(\frac{d}{dx}-\lambda \right)\left(\frac{d}{dx}-\mu \right)y=0
\end{eqnarray}
とかける。この式は$\forall y$で成立するので、
\begin{eqnarray}
\left(\frac{d}{dx}-\lambda\right)y=0\\
\end{eqnarray}
または、
\begin{eqnarray}
\left(\frac{d}{dx}-\mu\right)y=0
\end{eqnarray}
それぞれの微分方程式の解は、
\begin{eqnarray}
y=C_1e^{\lambda x},y=C_2e^{\mu x}
\end{eqnarray}
従って、重ね合わせの原理より、一般解は
\begin{eqnarray}
y=C_1e^{\lambda x}+C_2e^{\mu x}
\end{eqnarray}
($C_1,C_2$は任意定数)
\begin{itembox}[l]{例題2}
次の微分方程式を解け。
\begin{eqnarray}
y''-3y'+2y=0
\end{eqnarray}
\end{itembox}
解)\\
固有方程式
\begin{eqnarray}
s^2-3s+2=(s-2)(s-1)=0\\
s=1,2\\
y=C_1e^x+C_2e^{2x}
\end{eqnarray}
($C_1,C_2$は任意定数)\\
・$\lambda = \mu$(重根)の場合\\
\begin{eqnarray}
y=C(x)e^{\lambda x}
\end{eqnarray}
として、定数変化法を用いると、
\begin{eqnarray}
(s-x)^2=0\\
\left(\frac{d}{dx}\right)^2C(x)e^{\lambda x}=0\\
\left(\frac{d}{dx}-x\right)\left(\left(C'(x)e^{\lambda x}+\lambda C(x)e^{2x}\right)-\lambda C(x)e^{\lambda x}\right)\\
=\left(\frac{d}{dx}\right)C'(x)e^{\lambda x}\\
=C''(x)e^{\lambda x}
\end{eqnarray}
この関係式が$\forall x$で成立するには
\begin{eqnarray}
C''(x)=0,C(x)=C_1(x)+C_2\\
y=(C_1x+C_2)e^{\lambda x}
\end{eqnarray}
\begin{itembox}[l]{例題2}
次の微分方程式を解け。
\begin{eqnarray}
y''+2y'+y=0
\end{eqnarray}
\end{itembox}
解)
\begin{eqnarray}
s^2+2s+1=(s+1)^2=0\\
y=(C_1x+C_2)e^{-x}
\end{eqnarray}
・$\lambda,\mu$が共役複素数の場合
\begin{eqnarray}
\lambda = \alpha+i\beta\\
\mu=\alpha -i\beta
\end{eqnarray}
とする。そのとき、
\begin{eqnarray}
y=C_1e^{(\alpha+i\beta)x}+C_2e^{(\alpha -i\beta)x}\\
=e^{\alpha x}(C_1e^{i\beta x}+C_2e^{-i\beta x})
\end{eqnarray}
\begin{displaymath}
\left\{
\begin{array}{l}
e^{i\beta x}=\cos \beta x + i\sin \beta x  \\
e^{i\beta x}=\cos \beta x - i\sin \beta x
\end{array}
\right.
\end{displaymath}
より、
\begin{eqnarray}
XX=e^{\alpha x}(C_1(\cos \beta x + i\sin \beta x)+C_2(\cos \beta x-i\sin \beta x))\\
=e^{\alpha x}((C_1+C_2)\cos \beta x+i(C_1-C_2)\sin \beta x)\\
=e^{\alpha x}(C_1'\cos \beta x+C_2'\sin \beta x)\\
(C_1'=,C_2'=)
\end{eqnarray}
\begin{itembox}[l]{演習2}
次の微分方程式を解け。
\begin{eqnarray}
y''-4y'+13y=0
\end{eqnarray}
\end{itembox}
解)
\begin{eqnarray}
s^2-4s+13=0\\
s=2\pm 3i\\
y=e^{2x}(C_1\cos 3x+C_2 \sin 3x)
\end{eqnarray}
・定数係数の2階線形微分方程式\\
\quad $\lambda , \mu$ が実数,$\lambda \neq \mu$\\
\begin{eqnarray}
y=C_1e^{2x}+C_2e^{\mu x}
\end{eqnarray}
\quad $\lambda , \mu$ が実数,$\lambda = \mu$(重根)\\
\begin{eqnarray}
y=(C_1x+C_2)e^{2x}
\end{eqnarray}
\quad $\lambda , \mu$ が共役複素数,$\alpha \pm i\beta$
\begin{eqnarray}
y=e^{\alpha x}(C_1 \cos \beta x+C_2 \sin \beta x)
\end{eqnarray}
\\
・物理系への応用(初期値問題)
\begin{itembox}[l]{例題3}
長さ$l$の単振子の運動方程式
\begin{eqnarray}
\frac{d^2\theta}{dt^2}=-\frac{g}{l}\theta
\end{eqnarray}
を$\theta$について解け。\\
ただし、$g$は重力加速度,$\theta = \theta_0$の位置で、\\
この振子を静かに放したとする。
\begin{eqnarray}
(\theta|_{t=0}=\theta_0,\frac{d\theta}{dt}|_{t=0}=0)
\end{eqnarray}
\end{itembox}
解)
\begin{eqnarray}
\frac{d^2\theta}{dt^2}+\frac{g}{l}\theta =0 \\
s^2+\frac{g}{l}=0,s=\pm i\sqrt{\frac{g}{l}}
\end{eqnarray}
よって、
\begin{eqnarray}
\theta =C_1\sin \sqrt{\frac{g}{l}}t+C_2\cos \sqrt{\frac{g}{l}}t\\
\theta(0)=C_2=\theta_0\\
\frac{d\theta(0)}{dt}=\sqrt{\frac{g}{l}}C_1=0,C_1=0\\
\theta= \theta_0\cos \sqrt{\frac{g}{l}}t
\end{eqnarray}
\begin{itembox}[l]{演習3}
ばね定数$k$のばねに質量$m$のおもりがつながっている。おもりを自然長の位置($x=0$)から$\alpha$だけ引っ張り、初速度$\beta$で$-x$方向に移動させた。おもりに対する運動方程式を立て、位置xの解を求めよ。
\end{itembox}
解)
\begin{eqnarray}
s^2+\omega_0^2=0,s=\pm i\omega_0\\
x=C_1\sin \omega_0t+C_2\cos \omega_0t\\
x'=C_1\omega_0\cos \omega_0t-C_2\omega_0\sin \omega_0t\\
x(0)=C_2=\alpha\\
x'(0)=C_1\omega_0=\beta\\
x=\alpha \cos \omega_0t+\frac{\beta}{\omega_0}\sin \omega_0 t
\end{eqnarray}
\clearpage
\section{2階以上の線形微分方程式:定係数の非同次形}
・非同次方程式
\begin{eqnarray}
y''+a_1y'+a_2y=q(x)
\end{eqnarray}
の解は$q(x)=0$の場合の解(同次方程式の解)$Y_1(x)$と非同次方程式の特殊解$Y_2(x)$の和で表される。すなわち、
\begin{eqnarray}
y=Y_1(x)+Y_2(x)
\end{eqnarray}
・非同次方程式の応用例(RLC回路)
\begin{eqnarray}
v(t)=L\frac{di(t)}{dt}+Ri(t)+\frac{q(t)}{C}\\
i(t)=\frac{dq(t)}{dt}
\end{eqnarray}
より、
\begin{eqnarray}
v(t)=\frac{d^2q(t)}{dt^2}+R\frac{dq(t)}{dt}+\frac{q(t)}{C}
\end{eqnarray}
・非同次方程式(特殊解の解法)\\
未定係数法、定数変化法、演算子法、ラプラス変換法\\
・未定係数法\\
式XXにおいて、$q(t)$の関数系に従い、次の関数を特殊解として与え、係数$A_1,A_2,A_3...A_n$と決定する。\\
1)$q(x)$が$n$次多項式のとき
\begin{eqnarray}
Y_2=A_nx^n+A_{n-1}x^{n-1}+…+A_2x^2+A_1x+A_0
\end{eqnarray}
または、
\begin{eqnarray}
Y_2=A_nx^{n+1}+A^{n-1}+…+A_2x^3+A_1x^2+A_0x
\end{eqnarray}
2)$q(x)=ke^{bx}$のとき
\begin{eqnarray}
Y_2=A_0e^{bx}
\end{eqnarray}
または
\begin{eqnarray}
Y_2=A_0xe^{bx}
\end{eqnarray}
3)$q(x)=k\cos mx + l\sin mx$
\begin{eqnarray}
Y_2=A_0\cos mx+A_1\sin mx
\end{eqnarray}
または、
\begin{eqnarray}
Y_2=x(A_0\cos mx+A_1\sin mx)
\end{eqnarray}
\begin{itembox}[l]{例題1}
次の微分方程式を解け。
\begin{eqnarray}
y''-3y'+2y=x
\end{eqnarray}
\end{itembox}
解)\\
同次方程式の解は
\begin{eqnarray}
s^2-3s+2=(s-1)(s-2)=0\\
s=1,2\\
Y_1=C_1e^{2x}+C_2e^x
\end{eqnarray}
特殊解を
\begin{eqnarray}
Y_2=A_1x+A_0
\end{eqnarray}
とおき、非同次方程式に代入すると、
\begin{eqnarray}
\frac{d^2Y_2}{dt^2}-3\frac{dY_2}{dt}+2Y_2=0-3A_1+2(A_1x+A_0)\\
=2A_1x+2A_0-3A_1=x\\
2A_1=1,2A_0-3A_1=2A_0-\frac{3}{2},A_0=\frac{3}{4}\\
Y_2=\frac{x}{2}+\frac{3}{4}\\
y=Y_1+Y_2=C_1e^{2x}+C_2e^x+\frac{x}{2}+\frac{3}{4}
\end{eqnarray}
\begin{itembox}[l]{演習1}
次の微分方程式未定係数法により解け。
\begin{eqnarray}
(1)y''-y'-2y=x\\
(2)y''-2y'+y=e^{2x}\\
(3)y''+y=2\sin x
\end{eqnarray}
\end{itembox}
解)\\
(1)\\
同次解は
\begin{eqnarray}
Y_1=C_1e^{2x}+C_2e^{-x}
\end{eqnarray}
特殊解は
\begin{eqnarray}
Y_2=A_1x+A_2
\end{eqnarray}
とおくと、
\begin{eqnarray}
A_1=-\frac{1}{2},A_2=\frac{1}{4}\\
Y_2=-\frac{x}{2}+\frac{1}{4}\\
y=C_1e^{2x}+C_2e^{-x}-\frac{x}{2}+\frac{1}{4}
\end{eqnarray}
(2)\\
同次解は
\begin{eqnarray}
Y_1=A_0e^{2x},A_0=1,Y_2=e^{2x}\\
\therefore y=(C_1x+C_2)e^x+e^{2x}
\end{eqnarray}
(3)\\
同次解は$S=\pm i$より、
\begin{eqnarray}
Y_1=C_1\cos x+C_2\sin x\\
Y_2=A_1\cos x+A_2\sin x
\end{eqnarray}
とおくと、
\begin{eqnarray}
-A_1\cos x-A_2\sin x+A_1\cos x +A_2 \sin x=0=\neq 2\sin x
\end{eqnarray}
今度は、
\begin{eqnarray}
Y_2=x(C_1\cos x+C_2\sin x)
\end{eqnarray}
とおけば、
\begin{eqnarray}
2(-A_1\sin x+A_2 \cos x)-x(A_1\cos x+A_2 \sin x)+x(A_1\sin x+A_2\sin x)\\
=-2A_1\sin x+2A_2\cos x = 2\sin x\\
\therefore A_1=-1,A_2=0,Y_2=-x\cos x\\
\therefore y=C_1\cos x+C_2\sin x-x\cos x
\end{eqnarray}
・$n$階非同次方程式への展開\\
(※特殊解を2階の場合と全く同じように仮定して求める)
\begin{itembox}[l]{例題2}
次の微分方程式を解け。
\begin{eqnarray}
y^{(3)}+y''-y'-y=e^x
\end{eqnarray}
\end{itembox}
解)\\
同次解は
\begin{eqnarray}
s^3+s^2-s-1=(s-1)(s^2+2s+1)\\
=(s-1)(s+1)^2=0\\
\therefore s=\pm 1 (s=-1 \quad is \quad double \quad root)
\end{eqnarray}
特殊解は
\begin{eqnarray}
Y_2=A_0e^x
\end{eqnarray}
とおくと、
\begin{eqnarray}
A_0e^x+A_0e^x-A_0e^x-A_0e^x=0\neq e^x
\end{eqnarray}
そこで、
\begin{eqnarray}
Y_2=A_0xe^x
\end{eqnarray}
とおくと、
\begin{eqnarray}
A_0(3e^x+xe^x)+A_0(2e^x+xe^x)-A_0(e^x+xe^x)-A_0xe^x\\
=4A_0e^x=e^x\\
A_0=\frac{1}{4},Y_2=\frac{1}{4}e^x\\
y=C_1e^x+C_2e^{-x}+C_3xe^{-x}+\frac{1}{4}e^x
\end{eqnarray}
・定数変化法(2階非同次方程式の特殊解法)
同次会$Y_1(x)$の任意定数$C_1,C_2$を関数$C_1(x),C_2(x)$に置き換えたものを特殊解$Y_2(x)$とする。\\
すなわち、
\begin{eqnarray}
Y_2=C_1(x)y_1(x)+C_2(x)y_2(x)
\end{eqnarray}
また、以下の条件から$C_1(x),C_2(x)$を求める。
\begin{displaymath}
\left\{
\begin{array}{l}
C_1'(x)y_1(x)+C_2'(x)y_2(x)=0  \\
C_1'(x)y_1'(x)+C_2'(x)y_2'(x)=q(x)
\end{array}
\right.
\end{displaymath}
・定数変化法の証明
\begin{eqnarray}
y=Y_2=C_1(x)y_1(x)+C_2(x)y_2(x)\\
y'=(C_1'(x)y_1(x)+C_1(x)y_1'(x))+(C_2(x)y_2(x)+C_2(x)y_2'(x))\\
=(C_1y_1+C_2'y_2)+(C_1y_1'+C_2y_2')=0
\end{eqnarray}
とすると、
\begin{eqnarray}
y''=(C_1'y_1'+C_2y_2')+(C_1y_1''+C_2y_2'')=q(x)
\end{eqnarray}
とすると、
\begin{eqnarray}
y''+a_1y'+a_2y\\
=(q(x)+C_1y''+C_2y_2'')+a_1(C_1y_1'+C_2y_2')+a_2(C_1y_1+C_2y_2)\\
=C_1\cdot 0+C_2\cdot 0+q(x)=q(x)
\end{eqnarray}
\begin{itembox}[l]{例題3}
例題1を定数変化法で解け。
\end{itembox}
\begin{eqnarray}
Y_1=C_1e^{2x}+C_2e^x
\end{eqnarray}
\begin{displaymath}
\left\{
\begin{array}{l}
C_1'e^{2x}+C_2'e^x=0  \\
C_1'\cdot 2e^{2x}+C_2'e^x=x
\end{array}
\right.
\end{displaymath}
\begin{eqnarray}
C_1'=xe^{-2x},C_2'=-xe^{-x}\\
C_1=\int xe^{-2x}dx\\
=-\frac{1}{2}e^{-2x}+\frac{1}{2}\int e^{-2x}dx\\
=-\frac{1}{2}xe^{-2x}-\frac{1}{4}e^{-2x}+D_1\\
C_2=\int -se^{-x}dx\\
=xe^{-x}+e^{-x}+D_2\\
Y_2=\left(-\frac{1}{2}xe^{-2x}-\frac{1}{4}e^{-2x}+D_1\right)e^{2x}+(xe^{-x}+e^{-x}+D_2)e^x\\
=\frac{1}{2}x+\frac{3}{4}+D_1e^{2x}+D_2e^x\\
y=Y_1+Y_2\\
=E_1e^{2x}+E_2e^x+\frac{x}{2}+\frac{3}{4}
\end{eqnarray}
\begin{itembox}[l]{演習2}
演習1を定数変化法で解け。
\end{itembox}
(1)
\begin{eqnarray}
s^2-s-2=(s-2)(s+1)=0\\
s=-1,2
\end{eqnarray}
よって、
\begin{eqnarray}
Y_1=C_1e^{-x}+C_2e^{2x}
\end{eqnarray}
\begin{displaymath}
\left\{
\begin{array}{l}
C_1'e^{2x}+C_2'e^{-x}=0  \\
C_1'\cdot 2e^{2x}+C_2'e^{-x}=x
\end{array}
\right.
\end{displaymath}
\begin{eqnarray}
C_1'=\frac{1}{3}xe^{-2x},C_2'=-\frac{1}{3}xe^x\\
C_1=\frac{1}{3}\int xe^{-2x}dx\\
-\frac{e^{-2x}}{12}(2x+1)+D_1\\
C_2=-\frac{1}{3}\int xe^{-x}dx =-\frac{e^x}{3}(x-1)+D_2\\
Y_2=\left(-\frac{e^{-2x}}{12}(2x+1)+D_1\right)e^{2x}+\left(-\frac{e^x}{3}(x-1)+D_2\right)e^{-x}\\
=-\frac{1}{12}(2x+1)-\frac{1}{3}(x-1)+D_1e^{2x}+D_2e^{-x}\\
y=C_1e^{2x}+C_2e^{-x}-\frac{x}{2}+\frac{1}{4}
\end{eqnarray}
(2)
\begin{eqnarray}
s^2-2s+1=(s-1)^2=0\\
s=1
\end{eqnarray}
よって、
\begin{eqnarray}
Y_1=(C_1x+C_2)e^x\\
=C_1xe^x+C_2e^x
\end{eqnarray}
\begin{displaymath}
\left\{
\begin{array}{l}
C_1'e^x+C_2'e^x=0  \\
C_1'(e^x+xe^x)+C_2'e^x=e^{2x}
\end{array}
\right.
\left\{
\begin{array}{l}
C_1'x+C_2'=0  \\
C_1'(x+1)+C_2'=e^x
\end{array}
\right.
\end{displaymath}
\begin{eqnarray}
C_1'=e^x,C_2'=-xe^x\\
C_1=e^x+D_1,C_2=-xe^x+e^x+D_2\\
Y_2=(e^x+D_1)xe^x+(-xe^x+e^x+D_2)e^x\\
=e^{2x}\\
y=C_1xe^x+C_2e^x+e^{2x}
\end{eqnarray}
(3)
\begin{eqnarray}
s^2+1=0,s=\pm i
\end{eqnarray}
よって、
\begin{eqnarray}
Y_1=C_1\cos x+C_2\sin x
\end{eqnarray}
\begin{displaymath}
\left\{
\begin{array}{l}
C_1'\cos x+C_2'\sin x=0  \\
C_1'(-\sin x)+C_2'\cos x=2\sin x
\end{array}
\right.
\end{displaymath}
\begin{eqnarray}
C_1'=-2\sin ^2x=\cos 2x-1\\
C_2'=2\sin x\cos x=\sin 2x\\
C_1=\sin x\cos x -x+D_1\\
C_2=-\frac{1}{2}(1-2\sin^2x)+D_2\\
Y_2=(\sin x\cos x -x)\cos x -\frac{1}{2}(1-\sin^2x)\\
=\sin s\cos^2x+\sin^2x\sin x-\frac{1}{2}\sin x -x\cos x\\
=\frac{1}{2}\sin x -x\cos x\\
y=C_1\cos x+C_2\sin x-x\cos x
\end{eqnarray}
\begin{itembox}[l]{演習3}
例題2を定数変化法で解け。
\end{itembox}
解)
\begin{eqnarray}
Y_2=C_1e^x+C_2e^{-x}+C_3xe^{-x}\\
C_1'e^x+C_2'e^{-x}+C_3'xe^{-x}=0\\
C_1'e^x-C_2'e^{-x}+C_3'(e^{-x}-xe^{-x})=0\\
C_1'e^x+C_2'e^{-x}+C_3'(-2e^{-x}+xe^{-x})=e^x
\end{eqnarray}
XX+2XX+XX,$C_1'=\frac{1}{4}$
\begin{displaymath}
\left\{
\begin{array}{l}
C_2'e^{-x}+C_3'xe^{-x}=-\frac{1}{4}e^x  \\
C_2'(-e^{-x})+C_3'(e^{-x}-xe^{-x})=-\frac{1}{4}e^x
\end{array}
\right.
\end{displaymath}
\begin{eqnarray}
C_3'=-\frac{e^{2x}}{2}\\
C_2'=e^x\left(-\frac{1}{4}e^x+\frac{e^{2x}}{2}\cdot xe^{-x}\right)\\
=\frac{e^{2x}}{4}(2x-1)\\
C_1=\frac{1}{4}x+D_1\\
C_2=\frac{1}{4}\int (2xe^{2x}-e{2x})dx\\
=\frac{1}{4}\left(\left(xe^{2x}-\int e^{2x}dx\right)-\int e^{2x}dx\right)\\
=\frac{1}{4}(xe^{2x}-\frac{1}{2}e^{2x}-\frac{1}{2}e^{2x})\\
=\frac{e^{2x}}{4}(x-1)+D_2\\
C_3=-\frac{1}{2}\int e^{2x}dx=-\frac{e^{2x}}{4}+D_3\\
Y_2=\frac{x}{4}e^x+\frac{e^{2x}}{4}(x-1)e^{-x}-\frac{e^{2x}}{4}xe^{-x}\\
=\frac{1}{4}(xe^x+xe^x-e^x-xe^x)\\
=\frac{e^x}{4}(x-1)\\
y=C_1e^x+C_2e^{-1}+C_3xe^{-x}+\frac{1}{4}xe^x
\end{eqnarray}
\clearpage
\section{ラプラス変換}
\begin{itembox}[l]{ラプラス変換 \footnote{定数変化法よりも効率良く計算できる}}
区間$0\leq t\leq \infty $で定義された関数$f(t)$に対し、
\begin{eqnarray}
\int_{0}^{\infty}e^{-st}f(t)dt=\lim_{T \to \infty}\int_{0}^{T}e^{-st}f(t)dt
\end{eqnarray}
の値が有限であるとき、これを$f(t)$のラプラス変換といい、
$\mathcal{L}(f)$あるいは$F(s)$で表す。すなわち、
\begin{eqnarray}
\mathcal{L}(f)=F(s)=\int_{0}^{\infty}e^{-st}f(t)dt
\end{eqnarray}
逆に、$F(s)$を$f(t)$に戻すことをラプラス逆変換といい、
\begin{eqnarray}
\mathcal{L}^{-1}(F)=f(t)
\end{eqnarray}
と表す。\\
※ラプラス逆変換は通常、ラプラス変換表を用いて機械的に行う。
\end{itembox}
ラプラス変換表はXX頁に載せる\\
\begin{itembox}[l]{例題1}
次の関数をラプラス変換せよ。
\begin{eqnarray}
(1)a(Constant) \quad
(2)\cosh at \quad
(3)\cos at
\end{eqnarray}
\end{itembox}
解)\\
(1)
\begin{eqnarray}
\mathcal{L}(a)=\int_{0}^{\infty}e^{-st}\cdot adt\\
=a\left[-\frac{1}{s}e^{-st}\right]_{0}^{\infty}\\
=\frac{a}{s}
\end{eqnarray}
(2)
\begin{eqnarray}
\mathcal{L}(\cosh at)=\mathcal{L}\left(\frac{e^{at}+e^{-at}}{2}\right)\\
=\frac{1}{2}(\mathcal{L}(e^{at})+\mathcal{L}(e^{-at}))\\
=\frac{1}{2}\left(\frac{1}{s-a}+\frac{1}{s+a}\right)\\
=\frac{s}{s^2-a^2}
\end{eqnarray}
(3)
\begin{eqnarray}
\mathcal{L}(\cos at)=\mathcal{L}\left(\frac{e^{iat}+e^{-iat}}{2}\right)\\
=\frac{1}{2}(\mathcal{L}(e^{iat})+\mathcal{L}(e^{-iat}))\\
=\frac{1}{2}\left(\frac{1}{s-ia}+\frac{1}{s+ia}\right)\\
=\frac{s}{s^2+a^2}
\end{eqnarray}
\begin{itembox}[l]{演習1}
次の関数をラプラス変換せよ。
\begin{eqnarray}
(1)e^{at} \quad
(2)\sinh at \quad
(3)\sin at
\end{eqnarray}
\end{itembox}
解)\\
(1)
\begin{eqnarray}
\mathcal{L}(e^{at})=\int_{0}^{\infty}e^{-st}\cdot e^{at}dt\\
=\left[-\frac{1}{s-a}e^{-(s-a)t}\right]_{0}^{\infty}\\
=\frac{1}{s-a}
\end{eqnarray}
(2)
\begin{eqnarray}
\mathcal{L}(\sinh at)=\frac{a}{s^2-a^2}
\end{eqnarray}
(3)
\begin{eqnarray}
\mathcal{L}(\sin at)=\frac{a}{s^2+a^2}
\end{eqnarray}
・収束域\\
例題1の(1)において、もし$s<0$だとすると$\mathcal{L}(a)=\infty$となり、ラプラス変換の値がなくなる。\\
そこで、ラプラス変換の存在する$s$の範囲を収束域といい、この場合は$s>0$が該当する。\\
\begin{itembox}[l]{線形法則}
\begin{eqnarray}
\mathcal{L}(af(x)+bg(x))=a\mathcal{L}(f(x))+b\mathcal{L}(g(x))
\end{eqnarray}
\end{itembox}
\begin{itembox}[l]{移動法則}
\begin{eqnarray}
\mathcal{L}(e^{at}f(t))=F(s-a)
\end{eqnarray}
ex)
\begin{eqnarray}
\mathcal{L}(e^{at}\cosh bt)=\frac{s-a}{(s-a)^2-b^2}
\end{eqnarray}
\centerline{($s\rightarrow s-a$)}
\begin{eqnarray}
\mathcal{L}(e^{at}\cos bt)=\frac{s-a}{(s-a)^2+b^2}
\end{eqnarray}
\begin{eqnarray}
\mathcal{L}(f(t-a))=e^{-as}F(s)
\end{eqnarray}
\end{itembox}
\begin{itembox}[l]{微分法則}
\begin{eqnarray}
\mathcal(f'(x))=sF(s)-f(0)\\
\mathcal(f'(x))=s^2F(s)-sf(0)-f'(0)
\end{eqnarray}
\end{itembox}
\begin{itembox}[l]{積分法則}
\begin{eqnarray}
\mathcal{L}\left(\int f(x)dx\right)=\frac{F(s)}{s}+\frac{1}{s}\left[\int f(x)dx\right]_{x=0}
\end{eqnarray}
\end{itembox}
\begin{itembox}[l]{例題2}
RL直列回路に対し、$T=0$のとき、$E$[V]の電池をつないだ。このときの電流$i$をラプラス変換を用いて求めよ。ただし、$t<0$において$i=0$とする。
\end{itembox}
解)
\begin{eqnarray}
L\frac{di}{dt}+Ri=E\\
L(sI(s)-i(0))+RI(s)=\frac{E}{s}
\end{eqnarray}
初期条件$i(0)=0$より、
\begin{eqnarray}
Ls(s)+RI(s)=(Ls+R)I(s)=\frac{E}{s}\\
I(s)=\frac{E}{s(Ls+R)}\\
=\frac{\frac{E}{L}}{s+\frac{R}{L}}\\
=\frac{E}{L}\left(\frac{1}{s}-\frac{1}{s+\frac{R}{L}}\right)\\
i(t)=\frac{E}{R}\left(1-e^{1\frac{R}{L}t}\right)
\end{eqnarray}
\begin{itembox}[l]{演習2}
例題2の回路をRC直列回路に置き換えたときの電流$i$をラプラス変換を用いて答えよ。ただし、$t<0$において$i=0$である。
\end{itembox}
解)
\begin{eqnarray}
I(s)\left(\frac{1}{Cs}+R\right)=\frac{E}{s}\\
I(s)=\frac{E}{\frac{1}{C}+Rs}\\
=\frac{\frac{E}{R}}{s+\frac{1}{CR}}\\
i(t)=\frac{E}{R}e^{-\frac{1}{CR}t}
\end{eqnarray}
\clearpage
\begin{table}[h]
 \caption{ラプラス変換表} 
 \label{table:SpeedOfLight}
 \centering
  \begin{tabular}{clll}
   \hline
   $f(t)$ & $F(s)$ &収束域\\
   \hline \hline
   $a$ & $\frac{a}{s}$  & $s>0$\\
   $t$ & $\frac{1}{s^2}$ & $s>0$\\
   $t^n$ & $\frac{n!}{s^{n+1}}$ &  $s>0$ \\
   $e^at$ & $\frac{1}{s-a}$ & $s>a$ \\
   $\cos at$ & $\frac{s}{s^2+a^2}$ & $s>0$ \\
   $\sin at$ & $\frac{a}{s^2+a^2}$ & $s>0$ \\
   $\cosh at$ & $\frac{s}{s^2-a^2}$ & $s>|a|$ \\
   $\sinh at$ & $\frac{a}{s^2-a^2}$ & $s>|a|$ \\
   \hline
  \end{tabular}
\end{table}
\end{document}
